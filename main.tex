% Preamble
\documentclass{rapportcs}

% Packages
\usepackage{lipsum}

% Document
\title{Mémoire Game Programming} 
\begin{document}

%----------- Informations du rapport ---------

\logoentreprise{logos/entreprise.png}

    \titre{Real-Time Rendering of Molten Glass with \\ Raymarching Technique on Point Clouds}

    \mention{Game Programming}
    \master{Option GPU}
    \trigrammemention{MGP2 GPU}

    \eleve{Félix Chevalier \\ Manon Méhalin \\ Maxence Retier}
    
    \dates{Septembre 2024 - Juin 2025}

% Informations tuteurs ISART
    \tuteuruniv{
        \textsc{Maël Addoum} \\
        m.addoum@isartdigital.com
    }

    \tuteurentreprise{
        \textsc{Sylvain CONTASSOT-VIVIER} \\
        sylvain.contassotvivier@loria.fr \\
    }

%----------- Initialisation -------------------

    \fairemarges %Afficher les marges
    \fairepagedegarde %Créer la page de garde
    
%----------- Abstract -------------------
    \vspace*{\stretch{1}}
    \begin{center}
        \begin{abstract}
        \end{abstract}
    \end{center}
    \vspace*{\stretch{1}}
    \newpage

%------------ Table des matières ----------------
    \begingroup % start a TeX group
    \color{blue}
    \renewcommand*\contentsname{Table of contents}
    \tabledematieres % Créer la table de matières
    \endgroup

%------------ Corps du rapport ----------------


%------------ Introduction ----------------

    \section{Introduction}
% Effacer les lignes suivantes et écrire le texte souhaité

    Glassblowing is an art form with deep cultural and historical roots, especially in France, where it represents centuries of craftsmanship and tradition. However, preserving this heritage poses modern challenges, such as the environmental impact of energy-intensive training methods and adapting these artisanal techniques to new technologies.

    The overarching goal of the LORIA (Laboratoire Lorrain de Recherche Informatique et ses Applications) project is to design a fully immersive augmented-reality experience that allows users to blow and shape their own molten glass to create various forms, such as vases or sculptures. Our challenge is to balance the realism of the interactions and the technical performance, paving the way for new possibilities in digital creation and virtual craftsmanship.

    Beyond its technical innovations, this project contributes to the preservation of glassblowing while addressing sustainability. By providing a virtual training tool, our project could reduce the reliance on furnaces for practice, significantly lowering the ecological footprint of the learning process.

    We are creating a real-time simulation of molten glass using raymarching techniques on 3D point clouds. Raymarching, a rendering method in which the renderer progressively marches along rays to detect surfaces, offers a highly flexible approach to representing complex physical deformations. In our case, raymarching allows us to dynamically model the surface of molten glass.

    This work departs from traditional mesh-based rendering. Instead, we employ implicit surfaces and raymarching, as they are well-suited for handling point cloud data, enabling a seamless integration of dynamic physical simulation with advanced visualization techniques. Despite the complexity of molten glass, which behaves as both a fluid and a solid depending on temperature and movement, our approach proposes a promising solution for realistic real-time rendering.

    \newpage
    \section{The methodology}

    \newpage
    \section{Results and discussion}

    \newpage
    \section{Conclusions, future prospects}

    \newpage
    \section{Bibliographical references}
   \cite{yang_raymarching_2024} by Floney Yang, which helped us understand collision techniques, though it does not address deformable materials like glass.

    GitHub repositories like \cite{hecomi_raymarching_2024} and \cite{koskimies_raymarching_2024}, which offer practical raymarching examples but lack real-time physical simulation.

    Adrian Biagioli's blog \cite{biagioli_raymarching_2024} and Nabil N. Mansour's article \cite{mansour_sdf_2024}, both of which provide excellent foundations but do not cover dynamic systems like ours.

    \cite{grega2015modelling} focuses on using Finite Element Method (FEM) models to optimize the control of glass manufacturing processes, such as bottle production. It addresses real-time control challenges, aiming to improve energy efficiency and product quality through advanced supervisory control systems.

    However, this study does not provide a solution to our problem. It primarily targets industrial control rather than real-time graphical visualization. The FEM models discussed are computationally intensive and unsuitable for interactive simulations. Additionally, the paper does not explore techniques like raymarching or implicit surfaces, which are crucial to our project’s goal of simulating molten glass dynamically and realistically.

    \newpage
    \bibliography{main} % This is required to be included
    \bibliographystyle{unsrt} % vous pouvez utiliser d'autres styles pour lister vos références (regardez sur internet)

    \newpage
\end{document}